\documentclass{article}
\usepackage{xcolor}

\title{2IMF30 System Validation Assignment}


\begin{document}

\section{Introduction}
The `Maeslantkering' is a system designed to protect the harbour of Rotterdam against storms in the North Sea. Our task is to re-design the software of one of its components, namely the locomobile. \textcolor{blue}{Some high-level description of the system}.

\section{System description}
Each locomobile has:
\begin{itemize}
    \item 3 motors
    \item 3 oil pressure pumps (1 per motor/engine)
    \item 3 position sensors
\end{itemize}
The output commands are:
\begin{itemize}
    \item start motor N inward/outward
    \item start oil pump N
    \item stop motorn N inward/outward
    \item stop oil pump N

    \item stopped moving
    \item started fine-positioning
    \item sensor error N
    \item motor error N

    \item open/close valves
    \item pump water out
\end{itemize}
The input messages are:
\begin{itemize}
    \item close/open/stop locomobile (received twice)
    \item position sensor N
    \item motor N was repaired
\end{itemize}

The locomobiles communicate with the central system called `BesW'


\section{Global requirements}
\begin{enumerate}
    \item When a command is received once, it should be executed.
    \item When a command is received twice, it should be executed only once.

    \item The system shall use at most 1 motor while it is fine-positioning.
    \item When the door is further than 1 meter away from its desired position
    while fine-positioning, the system shall move the door back to the desired
    position.

    \item When the 'open locomobile' command is received while the door is closed,
    the system shall open the door.
    \item When the door is opened, the system shall start fine-positioning and
    report 'started fine-positioning'.

    \item When the 'stop locomobile' command is received while the door is moving,
    the system shall stop moving the door.

    \item When the 'close locomobile' command is received while the door is open,
    the system shall close the door.
    \item When the door reaches the closed position after receiving the 
    'close locomobile' command, the system shall open the valves, 
    start fine-positioning, and report 'stopped moving' and 'started fine-positioning'.


    \item After the system starts an oil pump or motor, the system shall not start
    any other oil pump or motor within 3 seconds.
    \item When position sensor N has reported a different position than the other two
        position sensors for 3 seconds, the system shall report 'sensor error N'. \textcolor{blue}{Might change this to be non-parametric}
    \item When the system reports 'sensor error N' while another sensor is broken, the
    system shall stop all running motors and oil pumps. \textcolor{blue}{Might change this to be non-parametric}
    \item While at least 2 sensors are broken, the system shall not start any motor or
    oil pump.

    \item When none of the position sensors reports a different value within 10 seconds 
    after starting motor N, the system shall report 'motor error N'.
    \item When the system reports 'motor error N', it shall not start motor N or oil pump
    N again until it receives 'motor N was repaired'.
\end{enumerate}

\section{Interactions}
Below is a list of all interactions, along with a description of their meaning.

\begin{itemize}
    \item \texttt{open}:  Open the door.
    \item \texttt{close}: Close the door.
    \item \texttt{stop}:  Stop current action. This can mean one of two things:
        \begin{enumerate}
            \item Stop moving the door, even when it is not in a final position.
            \item Stop fine-positioning.
        \end{enumerate}

    \item \texttt{finished}: The door is in position.
    \item \texttt{startFinePos}: Indicates that the locomobile has started fine-positioning.
    \item \texttt{errorSensor}:  Indicates that the sensors need human intervention, because their values will not concur.
    \item \texttt{errorMotor}:  Indicates that all three motors are malfunctioning and need human intervention.
\end{itemize}

\section{Requirements with interactions}
This is a reformulation of the requirements to include the interactions.

\begin{enumerate}
    \item 
\end{enumerate}

\section{Architecture}
Architectuur van het systeem.

\section{Behaviour (mCRL2)}
The behaviour of the system in mCRL2.

\section{Verification}
Verify using the toolset that all requirements given in item 3 above are valid for the design
in mCRL2.


\end{document}
