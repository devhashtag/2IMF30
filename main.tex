\documentclass{article}

\title{2IMF30 System Validation Assignment}


\begin{document}

\section{Introduction}
Hier komt een korte introductie.

% TODO Maak requirement lijst zonder actions

\section{System description}
% Een deel hiervan moet naar de introductie, neem patient support system als voorbeeld
Each locomobile has:
\begin{itemize}
    \item 3 motors
    \item 3 oil pressure pumps (1 per motor/engine)
    \item 3 position sensors
\end{itemize}
The output commands are:
\begin{itemize}
    \item start motor N inward/outward
    \item start oil pump N
    \item stop motorn N inward/outward
    \item stop oil pump N

    \item stopped moving
    \item started fine-positioning
    \item sensor error N
    \item motor error N
\end{itemize}
The input messages are:
\begin{itemize}
    \item close/open/stop locomobile (received twice)
    \item position sensor N
    \item motor N was repaired
\end{itemize}

The locomobiles communicate with the central system called `BesW'


\section{Global requirements}
\begin{enumerate}
    \item The system shall use at most 1 motor while it is \textit{fine-positioning}.
    \item When the door is further than 1 meter away from its desired position while fine-positioning, the system shall move the door back to the desired position.

    \item When the {\lq}open locomobile{\rq} command is received while the door is closed and at least 1 motor is usable, the system shall start at least 1 motor outward.
    \item When the {\lq}open locomobile{\rq} command is received while the door is closed and no motor is usable, the system shall report a {\lq}motor error{\rq} for all 3 motors.

    \item When the door is opened, the system shall start \textit{fine-positioning} and
    report {\lq}started fine-positioning{\rq}.

    \item When the {\lq}stop locomobile{\rq} command is received while the door is moving, the system shall stop all motors and oil pumps.

    \item When the {\lq}close locomobile{\rq} command is received while the door is open and at least 1 motor is usable, the system shall start at least 1 motor inward.
    \item When the {\lq}close locomobile{\rq} command is received while the door is open and no motor is usable, the system shall report a {\lq}motor error{\rq} for all 3 motors.
    
    \item When the door reaches the closed position after receiving the 
    {\lq}close locomobile{\rq} command, the system shall start \textit{fine-positioning} and report {\lq}stopped moving{\rq} and {\lq}started fine-positioning{\rq}.

    \item After the system starts an oil pump or motor, the system shall not start
    any other oil pump or motor within 3 seconds. % De oil pump en motor actions (en waarschijnlijk ook de sensors) zouden N als parameter moeten hebben
    \item When position sensor N has reported a different position than the other two
    position sensors for 3 seconds, the system shall report {\lq}sensor error N{\rq}.
    \item When the system reports {\lq}sensor error N{\rq} while another sensor is broken, the
    system shall stop all running motors and oil pumps.
    \item While at least 2 sensors are broken, the system shall not start any motor or
    oil pump.

    \item When none of the position sensors reports a different value within 10 seconds 
    after starting motor N, the system shall report {\lq}motor error N{\rq}.
    \item When the system reports {\lq}motor error N{\rq}, it shall not start motor N or oil pump
    N again until it receives {\lq}motor N was repaired{\rq}.
\end{enumerate}

\section{Interactions}
Beschrijving van interacties.

\section{Architecture}
Architectuur van het systeem.

\section{Behaviour (mCRL2)}
The behaviour of the system in mCRL2.

\section{Verification}
Verify using the toolset that all requirements given in item 3 above are valid for the design
in mCRL2.


\end{document}
