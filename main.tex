\documentclass{article}

\title{2IMF30 System Validation Assignment}


\begin{document}

\section{Introduction}
Hier komt een korte introductie.

\section{System description}
Each locomobile has:
\begin{itemize}
    \item 3 motors
    \item 3 oil pressure pumps
    \item 3 position sensors
\end{itemize}
The output commands are:
\begin{itemize}
    \item start motor N inward/outward
    \item start oil pump N
    \item stop motorn N inward/outward
    \item stop oil pump N

    \item stopped moving
    \item started fine-positioning
    \item sensor error N
    \item motor error N

    \item open/close valves
    \item pump water out
\end{itemize}
The input messages are:
\begin{itemize}
    \item close/open/stop locomobile (received twice)
    \item position sensor N
    \item motor N was repaired
\end{itemize}

One of the two channels is redundant so that every command is received twice. Each of the three engines has its own oil pressure pump. The sensors measure the same thing, so 2 of them are also redundant. During `fijnpositioneren', only one motor is used to keep the door in place. The locomobile communicates with the central system called `BesW'


\section{Global requirements}
\begin{itemize}
    \item When the 'open locomobile' command is received while the door is closed,
    the system shall open the door.
    \item When the door is opened, the system shall start fine-positioning and
    report 'started fine-positioning'.
    \item While the door is filled with water or the valves are open, the system 
    shall not open the door.

    \item When the 'stop locomobile' command is received while the door is moving,
    the system shall stop moving the door within 10 seconds.

    \item When the 'close locomobile' command is received while the door is open,
    the system shall close the door.
    \item When the door reaches the closed position after receiving the 
    'close locomobile' command, the system shall open the valves, 
    start fine-positioning, and report 'stopped moving' and 'started fine-positioning'.

    \item After the system starts an oil pump or motor, the system shall not start
    any other oil pump or motor within 3 seconds.
    \item When position sensor N has reported a different position than the other two
    position sensors for 3 seconds, the system shall report 'sensor error N'.
    \item When the system reports 'sensor error N' while another sensor is broken, the
    system shall stop all running motors and oil pumps.
    \item While at least 2 sensors are broken, the system shall not start any motor or
    oil pump.

    \item When none of the position sensors reports a different value within 10 seconds 
    after starting motor N, the system shall report 'motor error N'.
    \item When the system reports 'motor error N', it shall not start motor N or oil pump
    N again until it receives 'motor N was repaired'.
\end{itemize}

\section{Interactions}
Beschrijving van interacties.

\section{Architecture}
Architectuur van het systeem.

\section{Behaviour (mCRL2)}
The behaviour of the system in mCRL2.

\section{Verification}
Verify using the toolset that all requirements given in item 3 above are valid for the design
in mCRL2.


\end{document}
